\chapter{Discussion and Conclusions}
\label{chap:conclusions}
\lhead{\emph{Discussion and Conclusions}}
% In this chapter, you should expand upon (and initially reflect upon) the discussion and conclusion of the research phase of the project. The expectation here is that you should discuss the results presented in the previous evaluation section of the project in their totality (i.e. as a whole) from which you will then draw clear conclusions both on the quantitative and qualitative aspects of the overall project. This chapter should be a about 2000 words long (5 pages of text - 1600 words of discussion and 400 words of conclusion). This may vary depending on quality. The conclusion section of this report should conclude the project.

% Some suggested sections (the nature of this chapter should be discussed in detail with your term 2 supervisor):

\section{Solution Review}
% Discuss how well your solution solves the problem, based on your results from the evaluation chapter.
The final implementation of NoteReader addresses the core problem identified at the beginning of the project: the lack of an integrated solution that treats both document reading and note-taking as first class features.

The completed NoteReader prototype delivers a focused and practical solution to the challenge of fragmented note-taking and document reading workflows.  

The application successfully provides a split-view interface that allows users to read documents and take notes simultaneously, preserving contextual relevance and reducing cognitive load.

The system support all targeted document types, with an architecture designed to expand support to further formats. It supports document page to note linking and session persistence. While features like search, tagging, and Git synchronization were not completed, the application architecture supports their future inclusion.

While certain requirements were deferred to later development cycles, the project acts as a solid proof of concept, and illustrates the value of the core features. 

\section{Project Review}
% Discuss how well you addressed the project, and what you might do differently if you were to do it again. Make sure to identify how you handled any problems that arose during the project. Identify key skills that you learnt during the project, and clearly describe how you applied these, and how you might apply them differently if you were to do a similar project.

The project was approached using an agile methodology, structured around iterative development in five adjusted sprints. Although the original plan envisioned six evenly distributed sprints, difficulties encountered, particularly in areas involving document rendering, required adjustments to the original strategy.

Key challenges included the immaturity of the Kotlin Multiplatform framework, limitations in third-party libraries, and the time-intensive nature of converting diverse document types while preserving author intent. 

If this project were to be undertaken again, earlier identification of technological limitations would have enabled a more realistic scope. A stronger focus on UI prototyping and automated testing could also have improved development efficiency and user confidence.

Throughout the project, a number of valuable skills were developed:

\begin{itemize}
    \item \textbf{Cross-platform development}: Learned how to structure shared logic in Kotlin Multiplatform, and abstract platform-specific implementations.
    \item \textbf{Interface design and user experience}: Applied UX design principles to build an intuitive and minimalist interface.
    \item \textbf{Document processing and conversion}: Gained hands-on experience with libraries such as Apache POI, PDFBox, and external tools for document conversion.
    \item \textbf{State persistence and modular architecture}: Designed and implemented a maintainable file structure and session recovery mechanism.
\end{itemize}

These skills not only contributed to the successful delivery of NoteReader’s MVP but will be directly transferable to future software development projects that require cross-platform deployment, structured UI design, or file format interoperability.


\section{Conclusion}
% Enumerate the main conclusions you have got in terms of background, problem description and the solution approach you have come up with. Detail your primary and any secondary conclusions from your project.

This project set out to explore whether a unified application could bridge the gap between document reading and note-taking in a way that supports better organization, learning, and review. The investigation into current market tools revealed a recurring theme: fragmentation, platform lock-in, and poor contextual linking between notes and source materials.

NoteReader demonstrated a practical solution that merges document viewing and structured note-taking into a single interface. Core features, including split view layout, page specific note linking, plain text file storage, and persistent session state   were successfully implemented and validated through manual testing.

The primary conclusion is that the design approach enabled the delivery of a functional MVP with potential for further growth. 

Secondary, adopting cutting-edge frameworks (e.g., Kotlin Multiplatform) can accelerate innovation, but must be balanced against the increased risk of instability and limited ecosystem support.

NoteReader stands as a proof of concept that contextual, cross linked note taking is both feasible and desirable, and can serve to save time, prevent data loss and reduce cognitive load. 

\section{Future Work}
% Discuss any proposals for completion of the project, or for enhancements, or for re-design of your solution or software. Enumerate all the things you would have wanted to do should you have more time to work on this project.

While the project achieved a core Minimum Viable Product, there are several areas identified for further development and enhancement:

\begin{itemize}
    \item \textbf{Search and Tagging System}: Implementation of a tag-based and full-text search system for fast filtering of notes and documents remains a high-priority feature.
    \item \textbf{Git Integration}: Enable Git-based version control and synchronization to allow for cross-device usage and version history tracking of notes.
    \item \textbf{Cross-Platform Expansion}: Extend support to macOS, Linux, and mobile platforms using Kotlin Multiplatform’s shared logic model and Compose Multiplatform's UI capabilities.
    \item \textbf{OCR and Handwriting Support}: Add optical character recognition for image-based documents and enable stylus input for users who prefer handwritten notes.
    \item \textbf{User Customization}: Provide theme support, customizable keyboard shortcuts, and user settings for layout and accessibility.
    \item \textbf{Searchable Catalogue UI}: Build a graphical catalogue browser for easier navigation and visualization of stored documents and notes.
    \item \textbf{Improved EPUB Rendering}: Enhance rendering fidelity for EPUB files using more performant native libraries or better conversion processes.
    \item \textbf{Audio/Video Notes with Timestamps}: Allow users to take notes linked to specific timestamps during lecture recordings or media playback.
\end{itemize}

If more time and resources were available, these improvements would bring NoteReader closer to a fully featured application suitable for release and daily academic use. The current prototype serves as a solid foundation upon which these features can be gradually layered in a modular and maintainable way.